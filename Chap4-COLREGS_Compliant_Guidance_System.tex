\chapter{COLREGS Compliant Guidance System}
\label{chap:4_COLREGS_Compliant_Guidance_System}

\section{System Architecture}
\section{Global Guidance}
\section{Local Guidance}
VO because the algorithms collect all the velocities that result in collisions and present a set of collision-free velocities for human/machine, which facilitates human/machine to search for the best option.

The advantages of the VO algorithms have been noticed by researchers in maritime engineering. The idea of VO algorithms had ap- peared in the 1980s, named as Collision Threat Parameter Area (CTPA) (Degre and Lefevre, 1981; Lenart, 1983). Subsequently, Pedersen et al. (2003) showed that this method can provide a better support for the Officer On Watch (OOW) in collision prevention comparing with tra- ditional Automatic Radar Plotting Aid (ARPA). Later on, a series of studies proposed to use VO/CTPA algorithms for collision avoidance in various scenarios, e.g. restricted waters (Szlapczynski and Szlapczynska, 2017), multiple-ship (Szlapczynski, 2008), incorporating with regulations (Zhao et al., 2016), and unmanned ship (Kuwata et al., 2014). In (Huang et al., 2018), the algorithms which presume the target-ship keeps constant velocity, are concluded as a special case of the VO algorithm, called linear VO (LVO) algorithm. Details about the existing applications of VO algorithms in the maritime domain are addressed in Section 2.3.
