\chapter{Simulation and Results}
\label{chap:5_Simulation_And_Results}
        
    % Grammarlly: 95/100
    In this chapter, we present simulation scenarios for qualitative and quantitative evaluation of our system, as well as the experimental results. The simulations cover four different encounter scenarios between two vessels (head-on, crossing from the right, crossing from the left, and overtaking). For each encounter, we apply three variations, the first one uses our solution, the second does not use our solution, and the third uses our solution and is exposed to wind. For qualitative evaluation, we analyzed the final trajectories, avoidance success, and the behavior of our system when exposed to different environmental conditions. For quantitative evaluation, we analyzed the time for computation of our path planning method and the minimum distance kept between our \ac{COLREGS}-compliant vessel and the encountering vessel during the simulation.

    \section{Simulations Characterization}
    
    % Grammarlly: 100/100
    We run our simulations on USV\_sim\footnote{https://github.com/disaster-robotics-proalertas/usv\_sim\_lsa}~\cite{Paravisi2018Toward} simulator using the platform described in Table \ref{tab:simulation_platform_description}. In our simulations, we use a differential boat - shown in Figure \ref{fig:diffboat} - with two thrusters, which enables it to rotate over its axis. This boat is modeled according to specifications of the Lutra Prop boat, acquired from Platypus ~\cite{PlatypusLLC}. Beyond the specifications shown in Table \ref{tab:diffboat_specs}, the Lutra boat we use in our simulation has a laser rangefinder for environment scanning in its bow. We set the rangefinder to be capable of detecting objects within 25 meters in a range of 360°.
    \taburowcolors[1] 1{tableLineOne .. tableLineTwo}
    \tabulinesep = ^3mm_2mm
    \everyrow{\tabucline[.4mm  white]{}}

    \begin{table}[H]
        \caption{Simulation Platform Specification}
        \centering
            \begin{tabu} to \textwidth { >{\bfseries}X[c, 0] X[c, 3]}
            \tableHeaderStyle
            Component & Specification \\
            Computer & Desktop Dell XPS 8700 \\
            Processor & Intel® Core™ i7-4770 CPU @ 3.40GHz × 8 \\
            Memory & \begin{tabular}[c]{@{}l@{}}Teikon PC3-12800u DDR3 1600 MHz 2GB x 2\\ Teikon PC3-12800u DDR3 1600 MHz 4GB x 2\end{tabular} \\
            Operating System & Ubuntu 16.04.6 LTS \\
            ROS Version & ROS Kinetic
            \end{tabu}  
        \label{tab:simulation_platform_description}
    \end{table}
    
    % \taburowcolors[1] 1{tableLineOne .. tableLineTwo}
    % \tabulinesep = ^3mm_2mm
    % \everyrow{\tabucline[.4mm  white]{}}
    % \begin{table}
    %     \caption{Lutra Prop parameters}
    %     \centering
    %         \begin{tabu} to 0.45\textwidth { >{\bfseries}X[c, 2] X[c, 2]}
    %         \tableHeaderStyle
    %         Parameter       & Value       \\
    %         Length          & 106 cm      \\
    %         Width           & 48 cm       \\
    %         Height          & 15 cm       \\
    %         % Hull Volume     & ~0.02 $m^3$ \\
    %         Weight          & 9.7 Kg      \\
    %         % Extra Payload   & 3 Kg        \\
    %         % Thruster Force  & 22.54 N     \\
    %         % Linear drag     & 11.33       \\
    %         Maximum speed   & 1.41 m/s    \\
    %         \end{tabu}  
    %     \label{tab:simulation_platform_description}
    % \end{table}
    
    \begin{minipage}{\textwidth}
        \begin{minipage}[b]{0.35\textwidth}
        \centering
        \begin{figure}[H]
            \centering
            \includegraphics[scale=0.75]{figs/Chap5/diffboat.png}
            % \caption{Simulated version of Lutra Prop boat}
            % \label{fig:diffboat}
        \end{figure}
        \captionof{figure}{Simulated version of Lutra Prop boat}
        \label{fig:diffboat}
        \end{minipage}
    %   \hfill
        \begin{minipage}[b]{0.5\textwidth}
        \centering
            \begin{tabular}{cc}
                \toprule
                    \textbf{Parameter}       & \textbf{Value}       \\
                \midrule
                    Length          & 106 cm      \\
                    Width           & 48 cm       \\
                    Height          & 15 cm       \\
                    Weight          & 9.7 Kg      \\
                    Maximum speed   & 1.41 m/s    \\ 
                \bottomrule
            \end{tabular}
        \captionof{table}{Lutra Prop parameters}
        \label{tab:diffboat_specs}
        \end{minipage}
    \end{minipage}
    
    \vskip 1cm

    % Grammarlly: 100/100
    We assembled the scenarios in a simulated version of the Dilúvio stream. The Dilúvio stream is a potential location for real-world trials of our system since it is near our laboratory, so we evaluate the behavior of our system on a simulated version of it. The maintainers of the USV\_sim created this scenario. It was built by combining geographic location and relief information with geometry information and 3D buildings. This scenario simulates a real area of 1340mx555m. In Figure \ref{fig:simulation_diluvio_googleLocation2_roundedArea} we show with a red box the specific location\footnote{Google maps location: (-30.047258°, -51.232660°), Av. Edvaldo Pereira Paiva, 1970 - Praia de Belas - Porto Alegre - RS - Brazil} we simulated in our experiments and its simulated version.
     
    \begin{figure}
    \centering
        \begin{subfigure}[b]{0.48\textwidth}
            \centering
            \includegraphics[scale=0.12]{figs/Chap5/simulation_diluvio_googleLocation2_1_roundedArea.png}
            \caption{Real World Location}
            \label{fig:simulation_diluvio_googleLocation2_1_roundedArea}
        \end{subfigure}
        \hfill
        \begin{subfigure}[b]{0.48\textwidth}
            \centering
            \includegraphics[scale=0.122]{figs/Chap5/simulation_diluvio_googleLocation2_2_roundedArea.png}
            \caption{Simulated version of Real World Location }
            \label{fig:simulation_diluvio_googleLocation2_2_roundedArea}
        \end{subfigure}
    
    \caption{Real world region and its simulated version}
    \label{fig:simulation_diluvio_googleLocation2_roundedArea}
    \end{figure}
    

    % Grammarlly: 100/100
    As done by several authors (\eg{} Larson \etal{}~\cite{Larson2006Autonomous}, Naeem \etal{}~\cite{Naeem2012COLREGS}, Campbell \etal{}~\cite{Campbell2013Automatic}, Naus~\cite{Naus2013Idea}), for evaluation of the \ac{COLREGS}-compliance of our system, we assembled 4 main encounter scenarios between two vessels (head-on, crossing from the right, crossing from the left, and overtaking). In our scenarios one of the vessels has our \ac{COLREGS}-compliant system (from now on referred as \acl{OV}\footnote{Same nomenclature used by Naeem~\etal{}~\cite{Naeem2011Evasive}, He~\etal{}~\cite{He2017}, and other authors.} (\ac{OV})) and the other vessel (from now on referred as \acl{AV} (\ac{AV})) don't have any \ac{COLREGS} compliance knowledge. The starting configuration of each vessel in each scenario is described in Table \ref{tab:simulation_scenarios_configuration_own_vessel} and shown in Figure~\ref{fig:simulation_uwsim_encounters}. Both head-on and overtaking scenarios, were assembled right below the bridge shown in Figure \ref{fig:simulation_diluvio_googleLocation2_roundedArea}, while the crossing scenarios were assembled far the bridge, due to the need of more space between the start positions.


\begin{table}
    \caption{Own Vessel and Approaching Vessel Encounter Scenarios Configuration}
    \centering
    \begin{tabular}{ccccc} 
    \toprule
    \multirow{2}{*}{\begin{tabular}[c]{@{}c@{}} \textbf{}\\\textbf{Encounter}\\\textbf{Type} \end{tabular}} & \multicolumn{2}{c}{\textbf{Own Vessel} }                                                                                                                                      & \multicolumn{2}{c}{\textbf{Approaching Vessel} }                                                                                                                               \\ 
    \cline{2-5}
                                                                                                            & \begin{tabular}[c]{@{}c@{}}\textbf{Initial}\\\textbf{ Pose}\\\textbf{ (m, m, º)} \end{tabular} & \begin{tabular}[c]{@{}c@{}}\textbf{Target}\\\textbf{ Position} \end{tabular} & \begin{tabular}[c]{@{}c@{}}\textbf{Initial}\\\textbf{ Pose}\\\textbf{ (m, m, º)} \end{tabular} & \begin{tabular}[c]{@{}c@{}}\textbf{Target}\\\textbf{ Position} \end{tabular}  \\ 
    \hline
     \textbf{Head-on}                                                                                       & (450, 107.5, 0)                                                                                & (480, 107.5)                                                                 & (461, 107.5, 180)                                                                              & (350, 107)                                                                    \\
    \textbf{Crossing Right}                                                                                 & (410, 105, 90)                                                                                 & (410, 133)                                                                   & (416, 120, 215)                                                                                & (404, 115)                                                                    \\
    \textbf{Crossing Left}                                                                                  & (410, 105, 90)                                                                                 & (410, 133)                                                                   & (404, 105, 315)                                                                                & (416, 115)                                                                    \\
    \textbf{Overtaking}                                                                                     & (450, 107.5, 0)                                                                                & (600, 107.5)                                                                 & (461, 107.5, 0)                                                                                & (600, 107)                                                                    \\
    \bottomrule
    \end{tabular}
    \label{tab:simulation_scenarios_configuration_own_vessel}
\end{table}
    
    \begin{figure}[H]
    \centering
    
        \begin{subfigure}[b]{0.495\textwidth}
            \centering
            \includegraphics[width=\textwidth]{figs/Chap5/simulation_uwsim_headon_starting_pos.png}
            \caption{Head-on Encounter.}
            \label{fig:simulation_uwsim_headon_starting_pos}
        \end{subfigure}
        \begin{subfigure}[b]{0.495\textwidth}
            \centering
            \includegraphics[width=\textwidth]{figs/Chap5/simulation_uwsim_crossingright_starting_pos.png}
            \caption{Crossing from Right}
            \label{fig:simulation_uwsim_crossingright_starting_pos}
        \end{subfigure}
        
        \begin{subfigure}[b]{0.495\textwidth}
            \centering
            \includegraphics[width=\textwidth]{figs/Chap5/simulation_uwsim_crossingleft_starting_pos.png}
            \caption{Crossing from Left}
            \label{fig:simulation_uwsim_crossingleft_starting_pos}
        \end{subfigure}
        \begin{subfigure}[b]{0.495\textwidth}
            \centering
            \includegraphics[width=\textwidth]{figs/Chap5/simulation_uwsim_overtake_starting_pos.png}
            \caption{Overtaking}
            \label{fig:simulation_uwsim_overtake_starting_pos}
        \end{subfigure}
    
    \caption{Encounter scenarios for evaluation. Scenarios adapted from Huang~\etal{}~\cite{Huang2019Generalized}.}
    \label{fig:simulation_uwsim_encounters}
    \end{figure}
    
    \section{Experiments Results}

        %%%%%%%%%%%%%%%%%%%%%%%%%%%%%%%%%%%%%%
        %% Intro to Experiments Results
        %
        %% Grammarlly: 93/100
        %% v2.0
        %%%%%%%%%%%%%%%%%%%%%%%%%%%%%%%%%%%%%%
        We qualitatively evaluate the behavior of our method in two different configurations. In the first configuration, we compare the behavior of the system with and without \ac{ATC}.  In the second configuration, we compare our system's behavior with \ac{ATC} with and without the influence of wind. We executed both scenarios for four types of possible encounters between the two vessels. For the quantitative evaluation of each scenario, we measured the computation time of every execution of our path planner, and minimal distance maintained between the vessels, as well as whether the collision avoidance was successful or not. In Table \ref{tab:results}, we summarize the collected results.
        
        \section{Head-on}

        %%%%%%%%%%%%%%%%%%%%%%%%%%%%%%%%%%%%%%%%%%%%%%%%%%%%%%%%%%%%%%%%%%%%%%%%%%%%%%%%%%%%%%%%%%%%%%%%%%%%%%%%%%%%%%%%%%%%%%%%%%%%%%%%%%%%%%%%%%%%%%%%%%%%%%%%
        %% Head-On w vs wo \ac{ATC}
        %
        %% Grammarlly: 99/100
        %%%%%%%%%%%%%%%%%%%%%%%%%%%%%%%%%%%%%%%%%%%%%%%%%%%%%%%%%%%%%%%%%%%%%%%%%%%%%%%%%%%%%%%%%%%%%%%%%%%%%%%%%%%%%%%%%%%%%%%%%%%%%%%%%%%%%%%%%%%%%%%%%%%%%%%%
        
        In Figure \ref{fig:plot_ho_w_vs_wo}, we present, comparatively, the final trajectory of two vessels in two executions of the same head-on scenario. For the \ac{OV}, the performed trajectory is guided by our local guidance system, while the \ac{AV} is following a path towards a given goal ahead. In the "ATC case" execution, our system is fully functional, in the "no ATC" execution we removed the ability of the planning system to generate virtual obstacles, which is the core of the \ac{ATC} method and partially responsible for the \ac{COLREGS}-Compliant path planning. In both runs, in the time interval from t0 until just before t1, \ac{OV} goes northeast, due to a static obstacle located from (450, 104) to (460, 104). 
        
        Until t1, \ac{OV} tends to distance itself from the static obstacle. At t1 for both executions of the scenario, \ac{AV} becomes noticeable at the \ac{OV} local cost map. From t1 onwards, \ac{OV} reacts differently to each scenario. We observe that even with the existence of a static obstacle to the south in its proximity, \ac{OV} in the \ac{ATC} case decides to avoid the encounter with the other vessel moving to its starboard side, featuring a \ac{COLREGS}-Compliant behavior. \ac{OV} in "no ATC" case, is influenced only by the existence of an obstacle in the south and at the encounter with \ac{AV}, performs a not \ac{COLREGS}-compliant trajectory.
        
        \begin{figure}
            \centering
            \includesvg[width=\textwidth]{figs/Chap5/plot_ho_w_vs_wo.svg}
            \caption{Comparison of vessel trajectories in \ac{ATC} case and no \ac{ATC} in a head-on encounter scenario. Start position for \ac{OV} and \ac{AV} are marked with "$\blacktriangleright$" and "$\blacktriangleleft$", while their goals are marked with "x". In gray, we present static obstacles. In blue, we show the trajectory done by the \ac{AV}. In orange, we show the trajectory of the \ac{OV} with our path planner. In yellow, we show the trajectory of the \ac{OV} without our path planner. We can observe that our \ac{ATC} A* method implied \ac{COLREGS} compliance, since on the detection of another vessel (at time $t_1$) it avoided collision going to its starboard side.}
            \label{fig:plot_ho_w_vs_wo}
        \end{figure}
        
        %%%%%%%%%%%%%%%%%%%%%%%%%%%%%%%%%%%%%%
        %% Head-On w vs wo \ac{ATC} Computational Time
        %
        %% Grammarlly: 99/100
        %%%%%%%%%%%%%%%%%%%%%%%%%%%%%%%%%%%%%%
        Figure \ref{fig:plot_ho_w_vs_wo_CT} shows the computational time measured in seconds over time. Our \ac{COLREGS}-Compliant \ac{ATC} A* method had a peak cost of 0.364s for path generation for this head-on scenario. As we can see, both \ac{ATC} and no \ac{ATC} cases have similar computational cost curves. This happens because most of the computational cost of our path planning method is related to our A* implementation. The \ac{ATC} method alone has low computational consumption. The worst-case scenario consists of filling a grid area of dimension $w \times m$, where $w$ is equals to \ac{OV}'s width (2 local cost map grid cells), and $m$ is the worst case 99. Over time the computational time reduces due to the proximity to the goal position.
        
        \begin{figure}[H]
            \centering
            \includesvg[width=\textwidth]{figs/Chap5/plot_ho_w_vs_wo_CT.svg}
            \caption{Computational time comparison between \ac{ATC} case versus no \ac{ATC} in a head-on encounter over time. The system achieves a peak cost of around 0.364s using \ac{ATC} and around 0.369 without \ac{ATC}.}
            \label{fig:plot_ho_w_vs_wo_CT}
        \end{figure}
        
        %%%%%%%%%%%%%%%%%%%%%%%%%%%%%%%%%%%%%%
        %% Head-On w vs wo Wind
        %
        %% Grammarlly: 99/100
        %%%%%%%%%%%%%%%%%%%%%%%%%%%%%%%%%%%%%%
        In Figure \ref{fig:plot_ho_w_vs_wind} we compare final trajectories for the same head-on scenario described in \ref{tab:simulation_scenarios_configuration_own_vessel}, now for two executions of the simulation with different wind influence, in both we use \ac{ATC} A*. "No wind case" shows the behavior of the \ac{OV} being influenced by no wind. "Wind case" shows the trajectory of our \ac{OV} being influenced by the wind with northeast direction and intensity of 2.0 m/s, indicated by an arrow. We can see the change in the trajectory of \ac{OV} and that it still maintains \ac{COLREGS}-compliance. \ac{AV} response to the wind is related to the standard control system used; it seems not to be capable of acting against the 2 m/s wind intensity. In this work, we do not evaluate the limitations of the \ac{AV}'s control system. We empirically determined a 2.0 m/s for evaluation of our system. In all simulations using 2.0 m/s for wind intensity, our system was capable of reacting and avoid the collision. With wind intensity greater than 2.0 m/s, our system was not capable of avoiding collision, being strongly influenced by wind. This behavior is related to our internal controller (inside local planner and presented in \ref{sec:chap4_control}). 
        
        \begin{figure}[H]
            \centering
            \includegraphics[width=\textwidth]{figs/Chap5/plot_ho_w_vs_wind.pdf}
            \caption{Comparison between the trajectories of the vessels in Wind case and no Wind case in a head-on encounter scenario, both cases use \ac{ATC} A*. Start position for \ac{OV} and \ac{AV} are marked with "$\blacktriangleright$" and "$\blacktriangleleft$", while their goals are marked with "x". For the Wind case, the direction of the wind is northeast, represented by an arrow. We can observe that our \ac{ATC} A* kept implying COLREGS compliance even under the influence of wind in this scenario.}
            \label{fig:plot_ho_w_vs_wind}
        \end{figure}
        
        %%%%%%%%%%%%%%%%%%%%%%%%%%%%%%%%%%%%%%
        %% Head-On w vs wo Wind Computational Time
        %
        %% Grammarlly: 99/100
        %%%%%%%%%%%%%%%%%%%%%%%%%%%%%%%%%%%%%%
        Figure \ref{fig:plot_ho_w_vs_wind_CT} shows the computational time measured in seconds over time for wind and no wind cases. We observe that the computational time remains stable and similar to the previously presented cases. The computational time does not seem to be related to the imposed wind influence for this simulation.
        \begin{figure}[H]
            \centering
            \includesvg[width=\textwidth]{figs/Chap5/plot_ho_w_vs_wind_CT.svg}
            \caption{Computational time comparison between Wind case versus no Wind in a head-on encounter. The system achieves a peak cost of around 0.364s using \ac{ATC} without wind and around 0.353s with wind.}
            \label{fig:plot_ho_w_vs_wind_CT}
        \end{figure}

        \section{Crossing from the Right}
        %%%%%%%%%%%%%%%%%%%%%%%%%%%%%%%%%%%%%%%%%%%%%%%%%%%%%%%%%%%%%%%%%%%%%%%%%%%%%%%%%%%%%%%%%%%%%%%%%%%%%%%%%%%%%%%%%%%%%%%%%%%%%%%%%%%%%%%%%%%%%%%%%%%%%%%%
        %% Crossing Right w vs wo \ac{ATC}. 
        %
        %% Grammarlly: 100/100
        %%%%%%%%%%%%%%%%%%%%%%%%%%%%%%%%%%%%%%%%%%%%%%%%%%%%%%%%%%%%%%%%%%%%%%%%%%%%%%%%%%%%%%%%%%%%%%%%%%%%%%%%%%%%%%%%%%%%%%%%%%%%%%%%%%%%%%%%%%%%%%%%%%%%%%%%
        
        In Figure \ref{fig:plot_cr_w_vs_wo}, we present the behavior of our system when the \ac{OV} encounter another vessel coming from the right side. In Figure \ref{fig:plot_cr_w_vs_wo}, we show the comparison between trajectories with and without \ac{ATC}. As we can see, our method implies \ac{COLREGS} compliance when avoiding the collision. Regarding the trajectories, at time $t_1$ for each of the scenarios, the \ac{OV} has different behaviors. From the start until $t_1$, for both scenarios, \ac{OV} goes to the northwest direction. From $t_1$, \ac{OV} detects \ac{AV} in its local cost map. In the scenario with our local planner, \ac{OV} decides to avoid collision going to its starboard side, featuring a \ac{COLREGS}-compliant behavior. While for the scenario where \ac{OV} do not have our local planner, it decides to keep going northwest.
        
        \begin{figure}
            \centering
            \includesvg[width=\textwidth]{figs/Chap5/plot_cr_w_vs_wo.svg}
            \caption{Comparison of vessel trajectories in \ac{ATC} case and no \ac{ATC} in a crossing from the right encounter scenario. Start position for \ac{OV} and \ac{AV} are marked with "$\blacktriangle$" and "$\blacktriangleleft$", while their goals are marked with "x". In blue, we show the trajectory done by the \ac{AV}. In orange, we show the trajectory of the \ac{OV} with our path planner. In yellow, we show the trajectory of the \ac{OV} without our path planner. We can observe that our \ac{ATC} A* method implied \ac{COLREGS} compliance, since on the detection of another vessel (at time $t_1$) it avoided collision going to its starboard side.}
            \label{fig:plot_cr_w_vs_wo}
        \end{figure}

        Figure \ref{fig:plot_cr_w_vs_wo_CT} shows the computational time measured in seconds over time. Our \ac{COLREGS}-Compliant \ac{ATC} A* method had a peak cost of 0.395s for path generation for this crossing from the right scenario. As we can see, both \ac{ATC} and no \ac{ATC} cases have similar computational cost curves.
        
        \begin{figure}[H]
            \centering
                \includesvg[width=\textwidth]{figs/Chap5/plot_cr_w_vs_wo_CT.svg}
                \caption{Computational time comparison between \ac{ATC} case versus no \ac{ATC} in a crossing from the right encounter over time. The system achieves a peak cost of around 0.395s using \ac{ATC} and around 0.390 without \ac{ATC}}
                \label{fig:plot_cr_w_vs_wo_CT}
        \end{figure}
        
        %%%%%%%%%%%%%%%%%%%%%%%%%%%%%%%%%%%%%%
        %% Crossing Right w vs wo \ac{ATC}. WIND
        %
        %% Grammarlly: 100/100
        %%%%%%%%%%%%%%%%%%%%%%%%%%%%%%%%%%%%%%
        
        \begin{figure}[H]
            \centering
            \includegraphics[width=\textwidth]{figs/Chap5/plot_cr_w_vs_wind.pdf}
            \caption{Comparison between the trajectories of the vessels in Wind case and no Wind case in a crossing from the right encounter scenario, both cases use \ac{ATC} A*. Start position for \ac{OV} and \ac{AV} are marked with "$\blacktriangle$" and "$\blacktriangleleft$", while their goals are marked with "x". For the Wind case, the direction of the wind is northeast, represented by an arrow. We can observe that our \ac{ATC} A* kept implying COLREGS compliance even under the influence of wind in this scenario.}
            \label{fig:plot_cr_w_vs_wind}
        \end{figure}
        
        In Figure \ref{fig:plot_cr_w_vs_wind} we compare final trajectories for the same crossing from the right scenario described in \ref{tab:simulation_scenarios_configuration_own_vessel}, now for two executions of the simulation with different wind influence, in both we use \ac{ATC} A*. "No Wind case" shows the behavior of the \ac{OV} being influenced by no wind. "Wind case" shows the trajectory of our \ac{OV} being influenced by the wind with northeast direction and intensity of 2.0 m/s, indicated by an arrow. We can see the change in the trajectory of \ac{OV} and that it still maintains \ac{COLREGS}-compliance. 
        
        %%%%%%%%%%%%%%%%%%%%%%%%%%%%%%%%%%%%%%
        %% Crossing Right w vs wo WIND Computational Time
        %
        %% Grammarlly: 99/100
        %%%%%%%%%%%%%%%%%%%%%%%%%%%%%%%%%%%%%%
        Figure \ref{fig:plot_cr_w_vs_wind_CT} shows the computational time measured in seconds over time for wind and no wind cases. We observe that the computational time remains stable and and similar to the previously presented cases. The computational time does not seem to be related to the imposed wind influence for this simulation.
        \begin{figure}
            \centering
            \includesvg[width=\textwidth]{figs/Chap5/plot_cr_w_vs_wind_CT.svg}
            \caption{Computational time comparison between Wind case versus no Wind in a crossing from the right encounter. The system achieves a peak cost of around 0.395s using \ac{ATC} without wind and around 0.402s with wind.}
            \label{fig:plot_cr_w_vs_wind_CT}
        \end{figure}
        
        %AMA descobriu pq sem vento a curva mais mais longa do q com vento? parece estar invertido. convem verificar pois vao perguntar isso. nao esta fazendo  sentido.
        % DJ: Tenho novos dados prontos coletados mas nao consegui tempo pra plota-los e arrumar direito. Os novos dados são mais coerentes com os parâmetros.
        
         %AMA pq a posicao final do EV eh diferente com e sem vento ?
         
        %AMA nao entendi. quem nao atinge o objetivo ? na figura parece q sim. Ainda nao tem uma explicacao pq com vento a curva ficou melhor q sem vento
        
        \section{Crossing from the Left}
        %%%%%%%%%%%%%%%%%%%%%%%%%%%%%%%%%%%%%%%%%%%%%%%%%%%%%%%%%%%%%%%%%%%%%%%%%%%%%%%%%%%%%%%%%%%%%%%%%%%%%%%%%%%%%%%%%%%%%%%%%%%%%%%%%%%%%%%%%%%%%%%%%%%%%%%%
        %% Crossing Left w vs wo \ac{ATC}. 
        %
        %% Grammarlly: 100/100
        %%%%%%%%%%%%%%%%%%%%%%%%%%%%%%%%%%%%%%%%%%%%%%%%%%%%%%%%%%%%%%%%%%%%%%%%%%%%%%%%%%%%%%%%%%%%%%%%%%%%%%%%%%%%%%%%%%%%%%%%%%%%%%%%%%%%%%%%%%%%%%%%%%%%%%%%
        
        \begin{figure}[H]
            \centering
            \includesvg[width=\textwidth]{figs/Chap5/plot_cl_w_vs_wo.svg}
            \caption{Comparison of vessels trajectories in \ac{ATC} case and no \ac{ATC} in a crossing from the left encounter scenario. Start position for \ac{OV} and \ac{AV} are marked with "$\blacktriangle$" and "$\blacktriangleright$", while their goals are marked with "x". In blue we show the trajectory done by the \ac{AV}. In orange we show the trajectory of the \ac{OV} with our path planner. In yellow we show the trajectory of the \ac{OV} without our path planner.}
            \label{fig:plot_cl_w_vs_wo}
        \end{figure}
        
        In Figure \ref{fig:plot_cl_w_vs_wo}, we present the behavior of our system when the \ac{OV} encounter another vessel coming from the left side. In Figure \ref{fig:plot_cl_w_vs_wo}, we show the comparison between trajectories with and without \ac{ATC}. In this situation, both systems perform similar behavior in both scenarios, once for this situation, the \ac{OV} is not responsible for collision avoidance. Even so, when \ac{AV} appears in the \ac{OV}'s local cost map, the \ac{OV} starts to avoid proximity with \ac{AV} and go towards its starboard side (right). 
        
        Figure \ref{fig:plot_cl_w_vs_wo_CT} shows the computational time measured in seconds over time. Our \ac{COLREGS}-Compliant \ac{ATC} A* method had a peak cost of 0.388s for path generation for this crossing from the right scenario. As we can see, both \ac{ATC} and no \ac{ATC} cases have similar computational cost curves. The fall we see in the collected data (\eg{}~between 0.6 and 0.8 minutes in ATC case) happens due to a missing valid A* goal. This happens due to error on conversion between global and local locations using ROS move\_base standard conversion function.
        
        \begin{figure}[H]
            \centering
                \includesvg[width=\textwidth]{figs/Chap5/plot_cl_w_vs_wo_CT.svg}
                \caption{Computational time comparison between \ac{ATC} case versus no \ac{ATC} in a crossing from the left encounter over time. The system achieves a peak cost of around 0.388s using \ac{ATC} and around 0.415s without \ac{ATC}}
                \label{fig:plot_cl_w_vs_wo_CT}
        \end{figure}
        
        %%%%%%%%%%%%%%%%%%%%%%%%%%%%%%%%%%%%%%
        %% Crossing Left w vs wo \ac{ATC}. WIND
        %
        %% Grammarlly: 100/100
        %%%%%%%%%%%%%%%%%%%%%%%%%%%%%%%%%%%%%%
        
        \begin{figure}[H]
            \centering
            \includegraphics[width=\textwidth]{figs/Chap5/plot_cl_w_vs_wind.pdf}
            \caption{Comparison between the trajectories of the vessels in Wind case and no Wind case in a crossing from the left encounter scenario, both cases use \ac{ATC} A*. Start position for \ac{OV} and \ac{AV} are marked with "$\blacktriangle$" and "$\blacktriangleright$", while their goals are marked with "x". For the Wind case, the direction of the wind is northeast, represented by an arrow.}
            \label{fig:plot_cl_w_vs_wind}
        \end{figure}
        
        In Figure \ref{fig:plot_cl_w_vs_wind} we compare final trajectories for the same crossing from the left scenario described in \ref{tab:simulation_scenarios_configuration_own_vessel}, now for two executions of the simulation with different wind influence, in both we use \ac{ATC} A*. "No Wind case" shows the behavior of the \ac{OV} being influenced by no wind. "Wind case" shows the trajectory of our \ac{OV} being influenced by the wind with northeast direction and intensity of 2.0 m/s, indicated by an arrow.
        
        %%%%%%%%%%%%%%%%%%%%%%%%%%%%%%%%%%%%%%
        %% Crossing Left w vs wo WIND Computational Time
        %
        %% Grammarlly: 99/100
        %%%%%%%%%%%%%%%%%%%%%%%%%%%%%%%%%%%%%%
        Figure \ref{fig:plot_cl_w_vs_wind_CT} shows the computational time measured in seconds over time for wind and no wind cases. We observe that the computational time remains stable and similar to the previously presented cases. The computational time does not seem to be related to the imposed wind influence for this simulation. The fall we see in the collected data (\eg{}~between 0.6 and 0.8 minutes in "ATC case," and between 0.5 and 0.8 minutes in "no ATC case,") happens due to a missing valid A* goal. This happens due to error on conversion between global and local locations using ROS move\_base standard function.
        
        \begin{figure}[H]
            \centering
            \includesvg[width=\textwidth]{figs/Chap5/plot_cl_w_vs_wind_CT.svg}
            \caption{Computational time comparison between Wind case versus no Wind in a crossing from the left encounter. The system achieves a peak cost of around 0.388s using \ac{ATC} without wind and around 0.400s with wind.}
            \label{fig:plot_cl_w_vs_wind_CT}
        \end{figure}
        
        \section{Overtaking}
        %%%%%%%%%%%%%%%%%%%%%%%%%%%%%%%%%%%%%%%%%%%%%%%%%%%%%%%%%%%%%%%%%%%%%%%%%%%%%%%%%%%%%%%%%%%%%%%%%%%%%%%%%%%%%%%%%%%%%%%%%%%%%%%%%%%%%%%%%%%%%%%%%%%%%%%%
        %% Overtaking w vs wo \ac{ATC}. 
        %
        %% Grammarlly: 100/100
        %%%%%%%%%%%%%%%%%%%%%%%%%%%%%%%%%%%%%%%%%%%%%%%%%%%%%%%%%%%%%%%%%%%%%%%%%%%%%%%%%%%%%%%%%%%%%%%%%%%%%%%%%%%%%%%%%%%%%%%%%%%%%%%%%%%%%%%%%%%%%%%%%%%%%%%%
        
        In Figure \ref{fig:plot_ov_w_vs_wo}, we present the behavior of our system when \ac{OV} encounters another vessel ahead and decides to overtake it. In Figure \ref{fig:plot_ov_w_vs_wo}, we show the comparison between trajectories with and without \ac{ATC}. In this situation, a virtual obstacle was created in the front of the \ac{AV} and prevent the \ac{OV} from going in its front. In this situation, both cases ("ATC" and "no ATC") have similar results, since the only restriction in the path planning is related to occupy a position in front of the vessel that is being overtaken (\ie{}~the \ac{AV}). The \ac{OV} avoids going to the front of the \ac{AV} until the last moment before achieving its goal location. After achieving its goal, the \ac{OV} receives a new goal from the mission planner.
        
        \begin{figure}[H]
            \centering
            \includesvg[width=\textwidth]{figs/Chap5/plot_ov_w_vs_wo.svg}
            \caption{Comparison of vessel trajectories in \ac{ATC} case and no \ac{ATC} in a crossing from the left encounter scenario. Start position for \ac{OV} and \ac{AV} are marked with "$\blacktriangleright$", while their goals are marked with "x". In blue, we show the trajectory done by the \ac{AV}. In orange, we show the trajectory of the \ac{OV} with our path planner. In yellow, we show the trajectory of the \ac{OV} without our path planner.}
            \label{fig:plot_ov_w_vs_wo}
        \end{figure}
        
        Figure \ref{fig:plot_ov_w_vs_wo_CT} shows the computational time measured in seconds over time. Our \ac{COLREGS}-Compliant \ac{ATC} A* method had a peak cost of 0.356s for path generation for this crossing from the right scenario. As we can see, both \ac{ATC} and no \ac{ATC} cases have similar computational cost curves. In Figure \ref{fig:plot_ov_w_vs_wo_CT}, we can see a high peak of computational time in the beginning, followed by low computational time until the end of this simulation. The high peak happens at the start of the simulation when the \ac{AV} appears in the \ac{OV}'s local cost map. After the \ac{OV}'s decides to go its starboard side, the path planner can often find a straight path towards the goal, enabling low computational time cost. 
        % \todo{why the falls?}
        
        \begin{figure}
            \centering
                \includesvg[width=\textwidth]{figs/Chap5/plot_ov_w_vs_wo_CT.png}
                \caption{Computational time comparison between \ac{ATC} case versus no \ac{ATC} in a overtaking encounter over time. The system achieves a peak cost of around 0.356s using \ac{ATC} and around 0.368s without \ac{ATC}}
                \label{fig:plot_ov_w_vs_wo_CT}
        \end{figure}
        
        %%%%%%%%%%%%%%%%%%%%%%%%%%%%%%%%%%%%%%
        %% Overtaking w vs wo \ac{ATC}. WIND
        %
        %% Grammarlly: 100/100
        %%%%%%%%%%%%%%%%%%%%%%%%%%%%%%%%%%%%%%
        
        \begin{figure}[H]
            \centering
            \includegraphics[width=\textwidth]{figs/Chap5/plot_ov_w_vs_wind.pdf}
            \caption{Comparison between the trajectories of the vessels in Wind case and no Wind case in a crossing from the left encounter scenario, both cases use \ac{ATC} A*. Start position for \ac{OV} and \ac{AV} are marked with "$\blacktriangle$" and "$\blacktriangleright$", while their goals are marked with "x". For the Wind case, the direction of the wind is northeast, represented by an arrow.}
            \label{fig:plot_ov_w_vs_wind}
        \end{figure}
        
        In Figure \ref{fig:plot_ov_w_vs_wind} we compare final trajectories for the same overtaking scenario described in \ref{tab:simulation_scenarios_configuration_own_vessel}, now for two executions of the simulation with different wind influence, in both we use \ac{ATC} A*. "No Wind case" shows the behavior of the \ac{OV} being influenced by no wind. "Wind case" shows the trajectory of our \ac{OV} being influenced by the wind with northeast direction and intensity of 2.0 m/s, indicated by an arrow. In the same way, as in the previously present scenario, the \ac{OV} was capable of avoiding the collision successfully.  The \ac{OV} avoids going to the front of the \ac{AV} until the last moment before achieving its goal location. After achieving its goal, the \ac{OV} receives a new goal from the mission planner.

        %%%%%%%%%%%%%%%%%%%%%%%%%%%%%%%%%%%%%%
        %% Overtaking w vs wo WIND Computational Time
        %
        %% Grammarlly: 99/100
        %%%%%%%%%%%%%%%%%%%%%%%%%%%%%%%%%%%%%%
        Figure \ref{fig:plot_ov_w_vs_wind_CT} shows the computational time measured in seconds over time for wind and no wind cases. In this simulation, we achieved a higher computational time peak. This situation happened, when disturbed by the wind, the \ac{AV} occupied positions that generate an overlapped position between the local A* goal and the created virtual obstacle (time interval from around 2.85 to 4.4 minutes, in Figure \ref{fig:plot_ov_w_vs_wind_CT}). In this situation, our system searches for another valid local goal considering the global path within the local cost map. However, once again, in often consecutive attempts, the local planner was unable to found a valid A* goal. Each of these faults triggered the use of the last valid velocity command generated in our system.
        
        \begin{figure}[H]
            \centering
            \includesvg[width=\textwidth]{figs/Chap5/plot_ov_w_vs_wind_CT.svg}
            \caption{Computational time comparison between Wind case versus no Wind in a crossing from the left encounter. The system achieved a peak cost of around 0.356s using \ac{ATC} without wind and around 0.851s with wind.}
            \label{fig:plot_ov_w_vs_wind_CT}
        \end{figure}
        
        %%%%%% %%%%% %%%%%% %%%%% %%%%%% %%%%%% %%%%%
        
        %% Grammarlly: 100/100
        In Table \ref{tab:results}, we summarize the results for the different simulations we have done. In simulation scenarios, we can see that the computational time peak stays in a range from 0.368 to 0.416 for "no \ac{ATC}" configuration, from 0.357 to 0.390 for "\ac{ATC}" configuration and from 0.355 to 0.8852 for "Wind" configuration. Computation time for no \ac{ATC} and \ac{ATC} are similar since our \ac{ATC} method has a low impact on the A* implementation. The influence of wind in the overtaking scenario led our system to an exhaustive search and highest computational time cost peak. 
        
        %AMA algumas das discussoes q eu reclamei q faltavam estao aqui. Note que eu mencionei ALGUMAS .... 
        % mas vc deve lembrar q o leitor vai ler em ordem. Assim, ele vai ficar com duvidas ate chegar nesse ponto. Nesses casos, vc deve mencionar que, la onde a explicacao tinha q aparecer inicialmente, vc vai discutir tal coisa a seguir.
        
        %% Grammarlly: 96/100
        The average time stays similar for each scenario, regardless of the configuration (no \ac{ATC}, \ac{ATC}, and Wind) for head-on, crossing from the right, and crossing from the left scenarios. While for the overtaking scenario, the computational time achieved low values due to a simple search ahead for the path planning but presented the highest computational a strong restriction in the search space, caused by an overlap between the created virtual obstacle and the local A* goal.
        
        In all cases presented, there is at least a 1.59\footnote{Ratio between maximum and average computational time for crossing from the left (worst measured ratio)} ratio between the maximum and average time, implying stability below the maximum computational time sampled. The high values reached occur when there is some considerable restriction in the search space, that is when the goal for the local A* conflicts with the current position of \ac{AV} or with the created virtual obstacle. For each scenario, we measured the minimum distance kept by the \ac{OV} from the \ac{AV}. In our simulations, our system kept at least a distance of 1.505 meters from \ac{AV} (see Table \ref{tab:results}). 
        
        %AMA e isso eh considerado safe ? de acordo c o que  ou quem ? importante discutir isso.
        %AMA note q na tabela em uma min distance de 0.7 m !!!
        %DJ: nao ha concordancia na literatura que eu li sobre oq eu pode ser considero "safe", eles medem e informam apenas.
        
        
        

        \begin{table}
        % \caption{Based on \cite{Lazarowska2017New, Singh2018Constrained, Agrawal2015COLREGS, Candeloro2017Voronoi, Svec2013Dynamics}, we made a quantitative evaluation of our planning system measuring computational cost and the minimum distance kept between the vessels during simulations.}
        \caption{Computational time, minimum distance and successfully avoidance for each variation we evaluated regarding the four main encounters described in the \ac{COLREGS}.}
        \label{tab:results}
        \centering
        \begin{tabular}{ccccccc} 
        \toprule
        \multirow{2}{*}{\begin{tabular}[c]{@{}c@{}}\textbf{Encounter}\\\textbf{Type} \end{tabular}}     & \multirow{2}{*}{\textbf{Case} } & \multicolumn{3}{c}{\textbf{Computational Time (s)} }          & \multirow{2}{*}{\begin{tabular}[c]{@{}c@{}}\textbf{Successful }\\\textbf{Avoidance }\end{tabular}} & \multirow{2}{*}{\begin{tabular}[c]{@{}c@{}}\textbf{Minimum }\\\textbf{Distance (m) }\end{tabular}}  \\ 
        \cmidrule[\heavyrulewidth]{3-5}
                                                                                                        &                                 & \textbf{Maximum} & \textbf{Average} & \textbf{Std. Variation} &                                                                                                    &                                                                                                     \\ 
        \toprule
        \multirow{3}{*}{\textbf{Head-On}}                                                               & \textbf{No ATC}                 & 0.369            & 0.136            & 0.061                   & Yes                                                                                                & 4.431                                                                                               \\
                                                                                                        & \textbf{ATC}                    & 0.364            & 0.129            & 0.056                   & Yes                                                                                                & 1.599                                                                                               \\
                                                                                                        & \textbf{Wind}                   & 0.355            & 0.131            & 0.060                   & Yes                                                                                                & 1.505                                                                                               \\ 
        \cline{2-7}
        \multirow{3}{*}{\begin{tabular}[c]{@{}c@{}}\textbf{Crossing}\\\textbf{from Right}\end{tabular}} & \textbf{No ATC}                 & 0.390            & 0.178            & 0.107                   & Yes                                                                                                & 5.414                                                                                               \\
                                                                                                        & \textbf{ATC}                    & 0.390             & 0.249            & 0.060                   & Yes                                                                                                & 3.264                                                                                               \\
                                                                                                        & \textbf{Wind}                   & 0.403            & 0.253            & 0.062                   & Yes                                                                                                & 3.739                                                                                               \\ 
        \cline{2-7}
        \multirow{3}{*}{\begin{tabular}[c]{@{}c@{}}\textbf{Crossing}\\\textbf{from Left}\end{tabular}}  & \textbf{No ATC}                 & 0.416                 & 0.235                 & 0.105                        &  Yes                                                                                                  & 7.018                                                                                                    \\
                                                                                                        & \textbf{ATC}                    & 0.389                 & 0.224                 & 0.091                        & Yes                                                                                                   & 6.274                                                                                                    \\
                                                                                                        & \textbf{Wind}                   & 0.400                 & 0.208                 & 0.102                        & Yes                                                                                                   & 5.707                                                                                                    \\ 
        \cline{2-7}
        \multirow{3}{*}{\textbf{Overtaking}}                                                            & \textbf{No ATC}                 & 0.368            & 0.006            & 0.030                   & Yes                                                                                                & 3.325                                                                                               \\
                                                                                                        & \textbf{ATC}                    & 0.357            & 0.018            & 0.022                   & Yes                                                                                                & 3.101                                                                                               \\
                                                                                                        & \textbf{Wind}                   & 0.852            & 0.039            & 0.073                   & Yes                                                                                                & 1.787                                                                                               \\
        \bottomrule
        \end{tabular}
        \end{table}
        %AMA discuta pq crossing from left tem tempos menores (max, avg, std var)
        %AMA discuta p o maximo da ultima linha eh tao maior q os outros